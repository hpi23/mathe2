\documentclass[a4paper,10pt]{article}

% Use the geometry package to set the DIN A4 standard
\usepackage[a4paper, left=10mm, right=10mm, top=10mm, bottom=10mm]{geometry}

% Use the German locale
\usepackage[ngerman]{babel}
\usepackage[utf8]{inputenc}
\usepackage[T1]{fontenc}

% Use a sans-serif font
\renewcommand{\familydefault}{\sfdefault}

% Optional: Use a more modern sans-serif font, such as Helvetica or Arial
\usepackage{helvet}

% Subtitle https://tex.stackexchange.com/a/50186/197779
\usepackage{titling}
\newcommand{\subtitle}[1]{%
  \posttitle{%
    \par\end{center}
    \begin{center}\large#1\end{center}}%
}

\title{Ihr Titel Hier}
\subtitle{Untertitel}
\author{Ihr Name}
\date{\today}

\setlength{\parindent}{0pt} % Set paragraph indentation to zero

\usepackage{enumerate,amsthm,xcolor,graphicx,amsmath,amssymb,latexsym,framed,algorithmicx,algorithm,tikz,stmaryrd,enumitem}
\usepackage{multicol}
\newcommand{\br}{\hfill\newline\noindent}
\newcommand{\N}{\mathbb{N}}
\newcommand{\natnum}{\mathbb{N}}
\newcommand{\R}{\mathbb{R}}
\newcommand{\realnum}{\mathbb{R}}
\newcommand{\Z}{\mathbb{Z}}
\newcommand{\integers}{\mathbb{Z}}
\newcommand{\modulo}{\mod}
\newcommand{\lt}{<}
\newcommand{\gt}{>}
\newcommand{\set}[2]{\left\{#1 \mid #2\right\}}
\renewcommand{\epsilon}{\varepsilon}
\newcommand{\bigOSymbol}{\mathcal{O}}

\begin{document}

\section{Basics}

\begin{multicols}{3}

\subsection{Logarithmusgesetze}

\begin{align*}
    \log_a(xy) &= \log_a(x) + \log_a(y) \\
    \log_a\left(\frac{x}{y}\right) &= \log_a(x) - \log_a(y) \\
    \log_a(x^r) &= r \log_a(x) \\
    \log_a(1) &= 0 \\
    \log_a(a) &= 1 \\
    \log_a(a^x) &= x \\
    a^{\log_a(x)} &= x \\
    \log_a(x) &= \frac{\log_b(x)}{\log_b(a)} \quad (\text{Basiswechsel})
\end{align*}

\subsection{Potenzgesetze}

\begin{align*}
    a^m \cdot a^n &= a^{m+n} \\
    \frac{a^m}{a^n} &= a^{m-n} \\
    (a^m)^n &= a^{m \cdot n} \\
    (ab)^n &= a^n \cdot b^n \\
    \left(\frac{a}{b}\right)^n &= \frac{a^n}{b^n} \\
    a^0 &= 1 \quad \text{für } a \neq 0 \\
    a^{-n} &= \frac{1}{a^n}
\end{align*}

\subsection{Wurzelgesetze}

\begin{align*}
    \sqrt[n]{a \cdot b} &= \sqrt[n]{a} \cdot \sqrt[n]{b} \\
    \sqrt[n]{\frac{a}{b}} &= \frac{\sqrt[n]{a}}{\sqrt[n]{b}} \\
    \sqrt[n]{a^m} &= a^{\frac{m}{n}} \\
    \sqrt[n]{a} \cdot \sqrt[m]{a} &= a^{\left(\frac{1}{n} + \frac{1}{m}\right)} \\
    \sqrt[n]{\sqrt[m]{a}} &= \sqrt[mn]{a}
\end{align*}

\textbf{Allaussagen 4.13}

$K_5$ nicht planar

$K_{3,3}$ nicht planar

Jeder planare Graph geradlinig

Fermat's großer Satz $\forall n\in\N_{\ge 3} \neg \exists a,b,c\in\N: a^n + b^n = c^n$

Goldbach'sche Vermutung jede gerade Zahl Summe zweier Primzahlen

\textbf{Strings}

$A^* = \{\varepsilon\} \cup \bigcup_{i=1}^\infty A^i$

\textbf{Minimal (dein pp) / Maximal (mein pp)}

Minimal: keine kleineren Elemente

Kleinstes: alle anderen Elemente größer

Jedes kleinste Element minimal, aber nicht jedes minimale Element das kleinste

\textbf{Gaußsche Summenformel}

$\sum_{i=1}^n \; i \; = \; \frac{n(n+1)}{2}$

\textbf{Bernoulli}

$(1+x)^n \geq 1+nx$

\textbf{Binomischer Lehrsatz}

$(a+b)^n = \sum_{i=0}^n \binom{n}{i} a^i b^{n-i}$

\section{Aussagenlogik}

\textbf{Junktorenregeln}

De-morgan'sche Regel I

$\neg (A \vee B) \equiv \neg A \wedge \neg B$

De-morgan'sche Regel II

$\neg (A \wedge B) \equiv \neg A \vee \neg B$

Distributivgesetz I

$(A \wedge B)\vee C \equiv (A \vee C) \wedge (B \vee C)$

Distributivgesetz II

$(A \vee B)\wedge C \equiv (A \wedge C) \vee (B \wedge C)$

Implikation

$A\rightarrow B \equiv \neg A \vee B$

Äquivalenz

$A\leftrightarrow B \equiv (A \rightarrow B) \wedge (B \rightarrow A)$

\textbf{Quantorlogik}

$\forall$-Zusammenfassung

$(\forall x \in X: \varphi(x)) \wedge (\forall x \in X: \psi(x)) \equiv \forall x \in X: (\varphi(x) \wedge \psi(x))$

$\exists$-Zusammenfassung

$(\exists x \in X: \varphi(x)) \vee (\exists x \in X: \psi(x)) \equiv \exists x \in X: (\varphi(x) \vee \psi(x))$

\textbf{Ringschluss}

$ \varphi_{n} \rightarrow \varphi_1 \wedge \forall i \in [1..n-1]: \varphi_i \rightarrow \varphi_{i+1}. $

\textbf{Verkettung von Eigenschaften}

$f \text{inj/surj/bij} \land g \text{inj/surj/bij} \implies f\circ g\text{inj/surj/bij}$

\textbf{Rechtsinverses}

$A$ und $B$ nicht-leere Mengen, sowie $f\colon A \rightarrow B$
$f$ inj $\iff$ Es gibt eine surjektive Funktion $g\colon B \rightarrow A$ so, dass für alle $a \in A$ gilt $g(f(a))=a$

\textbf{Linksinverses}

$f$ surj $\iff$ Es gibt eine injektive Funktion $g\colon B \rightarrow A$ so, dass für alle $b \in B$ gilt $f(g(b))=b$

\textbf{Kardinalität}

$f\colon A \rightarrow B$ injektiv, wenn $|A| \leq |B|$.

$f\colon A \rightarrow B$ surjektiv, wenn $|A| \geq |B|$.

$A \subseteq B$, so gilt $|A| \leq |B|$.

$|A| \neq |Pot(A)|$.

\textbf{Taubenschlagprinzip}

$|A| > |B|$, dann $f: A \to B$ nicht inj

\textbf{Ordnungsrelation}

Transitiv $a \leq b \land b \leq c \rightarrow a \leq c$

antisymmetrisch $a \leq b \land b \leq a \rightarrow a=b$

reflexiv $a \leq a$

Äquivalenz: symmetrisch $a = b \rightarrow b = a$

\section{Algebra}

Magma: Abgeschlossenheit $a \cdot a \in A$

Halbgruppe: Assoziativität $(a \cdot b) \cdot c = a \cdot (b \cdot c)$

Monoid: Neutrales Element $a \cdot e = a = e \cdot a$

Gruppe: Inverses Element $a \cdot b = e = b \cdot a$

\textbf{Gruppeneigenschaften Funktionen}

$(\mathcal{F}_A,\circ)$ Funktionen $A\to A$ Monoid

$(\mathcal{B}_A,\circ)$ Bijektionen $A\to A$ Gruppe

\textbf{Logische Operatoren}

$(\{\mathtt{t},\mathtt{f}\},\mathtt{AND})$ ist ein kommutatives Monoid mit neutralem Element $\mathtt{t}$, aber keine Gruppe.

$(\{\mathtt{t},\mathtt{f}\},\mathtt{XOR})$ ist eine kommutative Gruppe mit neutralem Element $\mathtt{f}$.

$(\{\mathtt{t},\mathtt{f}\},\mathtt{OR})$ ist ein kommutatives Monoid mit neutralem Element $\mathtt{f}$, aber keine Gruppe.

\textbf{Argumentweise Operation}

$(f \diamond g) (a) = f(a) \diamond g(a)$ erbt Struktureigenschaften von einzelner Operation.

Inverses / Neutrales ist eindeutig

$(ab)^{-1} = b^{-1} a^{-1}$

\textbf{Ring}

Dann ist $(A,+,\cdot)$ ein Ring falls $(A,+)$ eine kommutative Gruppe mit einem neutralen Element $0$ ist und $(A, \cdot)$ eine Halbgruppe

$\forall a,b,c \in A: (a+b)\cdot c = a\cdot c + b \cdot c$

$\forall a,b,c \in A: a\cdot(b+c) = a\cdot b + a \cdot c$

Körper: $(A \setminus \{0\},\cdot)$ eine kommutative Gruppe

\textbf{Ringrechenregeln}

Für alle $a \in R$, $0a = 0 = a0$.

Für alle $a,b \in R$, $a(-b) = -(ab)$ (mit anderen Worten: $ab$ ist das additiv Inverse von $a \cdot (-b)$).

Für alle $a,b \in R$, $(-a)b = -(ab)$.

Für alle $a,b \in R$, $(-a)(-b) = ab$.

Falls $(R,\cdot)$ ein neutrales Element $1$ hat, dann gilt, für alle $a \in R$, $-a = (-1)a$.

\textbf{Untergesocks}

Untermonoid muss gleiches neutrales Elemnt haben. Es kann Teilmengen mit anderem neutralem Element geben.

7.5 Neutrales Element Untergruppe gleich Gruppe.

\textbf{Untergruppenkriterium}

$(G,\cdot)$ Gruppe, $U\subseteq G$, $U\neq \emptyset$ $a,b\in U: a\cdot b^{-1}\in U$.

\textbf{Produktstruktur}

(kartesisches Gruppenprodukt) $(A,\diamond) \times (B,\ast) = (C,\otimes)$ Gruppe mit $(a,b) \otimes (a',b') = (a \diamond a',b \ast b')$

\textbf{Homomorphismus}

$f: A \to B$ mit $f(a \diamond b) = f(a) \ast f(b)$

wenn bijektiv: Isomorphismus.

Isomorphismus erhält Struktur.

Inverser Isomorphismus auch Isomorphismus.

\textbf{Zyklische Gruppe}

$\langle g \rangle = \set{g^i}{i \in \integers} = G$

$|\langle g \rangle|$ Ordnung von g

$\langle g \rangle$ abelsch, $\subseteq$-kleinste Untergruppe die g hat

\textbf{Potenzen von Gruppenelementen}

$(g^i)^{-1} = g^{-i}$, $g^i \cdot g^j = g^{i+j}$

Bei G Ordnung k: $g^{k-1} = g^{-1}$, k kleinste Zahl mit $g^k=1$

\textbf{Satz von Lagrange} $|U|\;|\;|G|$

\textbf{Kleiner Satz von Fermat}

$g^{|G|} = 1$, also auch $a^{-1} = a^{k-1}$. Für Primzahlen p ist $a^{p-1} \equiv_p 1$. Weiterhin lässt sich das Inverse zu $a$ in $\integers_p^*$ berechnen als $a^{p-2}$.

Sei $p$ eine Primzahl. Dann kann man mit $2 \log_2(p)$ Multiplikationen ein Inverses von $a \in \integers_p^*$ berechnen.

\textbf{Wurzel}

$r = \sqrt{g}$, falls $r^2 = g$

\textbf{Polynome}

$R[X]$ entspricht $x \mapsto \sum_{i=0}^d a_i x^i$, ist Ring wenn R Ring.

Multiplikation $c=a_0\cdots a_d \cdot b_0\cdots b_{d'}$: $c_n = \sum_{i=0}^n a_i b_{n-i}$

\textbf{Nullstellensatz}

$a$ Nullstelle $\implies$ $(x-a)$ teilt Polynom. $p(x)$ höchstens $grad(p)$ Nullstellen.

\section{Zahlentheorie}

\textbf{Modulo}

$a \mod m = r \equiv \exists x \in \Z: a = xm +r$

$m|a \equiv a \mod m = 0$

$am \modulo m = 0$

$(am+b) \modulo m = b \modulo m$

\textbf{Vertauschung}

$(a \modulo m) + (b \modulo m) \modulo m = (a+b) \modulo m$.

$(a \modulo m) \cdot (b \modulo m) \modulo m = (a \cdot b) \modulo m$

\textbf{Teilerfremd}

$a \ m \equiv \{x \in \integers\}{x |a \; \wedge \; x | m} = \{1,-1\}$

\textbf{Teilbarkeit durch 9}

$a \in \N$ mit Ziffern $a_i$

$a = \sum_{i=0}^k 10^i \cdot a_i$

Dann gilt $9 | a$ genau dann, wenn $9 | \sum_{i=0}^k a_i$

\textbf{Lemma von Euklid}

Seien $a,b$ und $m \in \natnum_+$ so, dass $a,m$ teilerfremd sind. Falls nun $m | ab$, so gilt $m|b$.
Insbesondere gilt für alle $p$ Primzahlen und $a,b \in \integers$, dass falls $p | ab$, so $p | a$ oder $p | b$. 

\textbf{Fundamentalsatz Arithmetik}

Sei $a \in \natnum_+$. Dann gibt es genau ein 
$k \in \natnum$, genau eine aufsteigende Sequenz an Primzahlen 
$p_1 \lt p_2 \lt \cdots \lt p_k$ und genau eine Sequenz an Exponenten $z_1,…,z_k$ so, dass $a = \Pi_{i=1}^k \; p_i^{z_i}.$ 

\textbf{Restklassen}

$\integers_m = \{0,…,m-1\}$ $\integers_m^* = \{1,…,m-1\}$

Sei $m \in \natnum_{\geq 2}$. Es gelten die folgenden Aussagen.

$(\integers_m,+_m)$ ist eine abelsche Gruppe.

Falls $m$ eine Primzahl ist, so ist $(\integers_m^*,\cdot_m)$ eine Gruppe.

$(\integers_m,+_m,\cdot_m)$ ist ein Ring, bei dem die multiplikative Halbgruppe ein kommutatives Monoid ist.

Falls $m$ eine Primzahl ist, so ist $(\integers_m,+_m,\cdot_m)$ ein Körper.

\textbf{Inverse Modulo}

Seien $a,m > 1$ teilerfremd. Dann gibt es ein $b \in \integers$ so, dass $ab \equiv_m 1$. 

\section{Graphen}

$G=(V,E)$

Schleife $(v,v)$

Pfad = Knotentupel, keine doppelten Knoten

Kantenzug = Kantentupel, keine doppelten Kanten

Einfacher Graph: ungerichtet, schleifenlos

Ausgangsgrad $\delta_G^+(v) = |N_G^+(v)|$

Eingangsgrad $\delta_G^-(v) = |N_G^-(v)|$

Durchmesser $\max \{d(v,w) | v,w \in V\}$

\textbf{Distanz}

11.5 nur Metrik, wenn ungerichtet und zusammenhängend

\textbf{Partit}

k-partit: disjunkte Aufteilung in k Knotenmengen ohne Kanten untereinander.

11.12: G bipartit $\iff$ jeder Kreis in G gerade Länge.

\textbf{Arten}

$K_n$: vollständiger Graph

$K_{n,m}$: vollständig bipartiter

$P_n$: Pfad

$C_n$: Kreis

$K_{1,n}$: Stern

\textbf{Teilgraphen}

$G'[U] = (U, \{(u,v) \in U^2 | (u,v) \in E'\})$ von U induzierter Teilgraph von G

\textbf{Isomorphie}

Bijektion $\varphi\colon V \rightarrow V'$ mit $\forall u,v \in V: (u,v) \in E \; \leftrightarrow \; (\varphi(u),\varphi(v)) \in E'$

\textbf{Operationen}

Graphen vereinigen: $G \uplus G'$ 

Richtung umdrehen $G^{-1}$

\textbf{Handschlaglemma}

$\sum_{v \in V} \; \delta^+(v) \; = \; |E| \; = \; \sum_{v \in V} \; \delta^-(v)$

Bei gerichteten Graphen $|E|$ ungerichtete Kantenanzahl:

$\sum_{v \in V} \; \delta(v) \; = \; |E| \; = \; 2| \{\{u,v\} | (u,v) \in E\}|$

\textbf{Euler}

Eulerpfad / -kreis: Kantenzug, jede Kante genau einmal erwischt, Knotendopplung ok (geht gerichtet / ungerichtet)

Ungerichteter G hat Eulerpfad $\equiv$ 2 oder 0 Knoten mit ungeradem Grad. Bei Eulerkreis 0 Knoten mit ungeradem Grad.

\textbf{Hamilton}

Hamiltonpfad / -kreis: Pfad, jeder Knoten genau einmal erwischt

\textbf{Binärbäume}

strikter: 2 oder 0 Kinder (12.7 ein Blatt mehr als innere Knoten)

vollständiger: strikter Binärbaum, alle Blätter gleiche Höhe

\textbf{Baumeinordnung}

G Baum $\equiv$ G maximal kreisfrei $\equiv$ G minimal zusammenhängend $\equiv$ G kreisfrei mit $|E|=n-1$ $\equiv$ G zusammenhängend mit $|E|=n-1$.

\subsection{Planare Graphen}

\textbf{Euler'sche Polyederformel}

Nicht-leerer, zusammenhängender planarer Graph mit n Knoten, m Kanten, f Flächen: $n+f-m = 2$

\textbf{Kantenbeschränkung} $m \leq 3n-6$

\textbf{Minimalknotengrad} ein Knoten Grad höchstens 5.

\textbf{Färbbarkeit} Immer 4, 5, 6-färbbar.

\textbf{Bipartite Kantenbeschränkung}

Bipartiter, zusammenhängender, planarer Graph mit $\ge3$ Knoten hat max $2n-4$ Kanten

\textbf{Nicht-Planar}

$K_5, K_{3,3}$ minimal nicht-planar.

\textbf{Fußballsatz}

schwarze Fünfecke, weiße Sechsecke, jede Ecke 3 Flächen $\implies$ 12 schwarze Stücke.

\section{Grenzwerte}

$\forall^\infty \equiv \exists n_0 \in \natnum \forall n \in \natnum_{\geq n_0}$

$\exists^\infty \equiv \forall n_0 \in \natnum \exists n \in \natnum_{\geq n_0}$

\textbf{Folgen-Grenzwert}

$lim_{n\to\infty} f(n) = r \iff \forall \varepsilon \in \realnum_+ \forall^\infty n \in \natnum: |f(n) - r| \leq \varepsilon$

\textbf{Beispiel 13.3}

$(c)_{n\in \natnum}$ Grenzwert c.

$((-1)^n)_{n \in \natnum}$ kein Grenzwert.

$(n)_{n \in \natnum}$ kein Grenzwert.

$(1/(n+1))_{n \in \natnum}$ Grenzwert 0.

\textbf{Limes-Rechenregeln}

Addition, Multiplikation, Kehrwert, Potenz, Wurzel, Betrag, Division legit

\textbf{Konvergenz des Betrags}

$\lim_{n\to\infty} |f(n)|=0 \implies \lim_{n\to\infty} f(n)=0$

\textbf{Teilfolge}

$f:\N\to\R$ Folge, $g:\N\to\N$ strikt mon. steigend, $f\circ g$ ist Teilfolge von $f$. $\lim_{n\to\infty}f(n)=r \iff \forall g: \lim_{n\to\infty} f\circ g = r$

\textbf{Grenzwert reeler Funktionen}

$\lim_{x\to x_0}f(x) = r \iff \forall \epsilon\in\R_+\; \exists \delta\in\R_+\; \forall x\in\R: 0 < |x_0-x| \le \delta \implies |f(x)-r| \le \epsilon$

Jeder $\epsilon$-Schlauch hat eine $\delta$-Säule

13.10 $\lim_{x \rightarrow x_0} (f+g)(x) = r+s$
$\lim_{x \rightarrow x_0} (f \cdot g)(x) = r \cdot s$

\textbf{Sandwich reelle Grenzwerte}

$f(x) \leq g(x) \leq h(x)$, dann $\lim_{x \rightarrow x_0} f(x) = \lim_{x \rightarrow x_0} h(x)=r \implies \lim_{x \rightarrow x_0} g(x)=r$

\subsection{Stetigkeit}

\textbf{Stetige Funktionen 13.13}

$c, x, x^2, 1/x, exp, ln$

\textbf{Argumentweise Addition/Multiplikation}

$f$, $g$ stetig $\implies$ $f \circ g$, $f+g$, $f\cdot g$ stetig

Polynomfunktionen immer stetig

\textbf{Folgen-Stetigkeit}

$f$ stetig $\iff$ $(x_n)_{n\in\N}$ konvergent $\implies$ $\lim_{n \rightarrow \infty} f(x_n) = f\left(\lim_{n \rightarrow \infty} x_n \right)$

\textbf{Zwischenwertsatz} Wenn $f$ stetig, muss jeder Zwischenwert erreicht werden

\textbf{Stetigkeitsarten}

Lipschitz-Stetig $\subseteq$ $\epsilon$-$\delta$-Stetig = Folgen-Stetig $\subseteq$ ZW-Stetig 

\section{O-Notation}

\textbf{Ordnung auf Folgen}

$f \leq g \Leftrightarrow \forall n \in \natnum: f(n) \leq g(n)$ nicht-totale Ordnungsrelation

\textbf{Schließliche Ordnung}

$f \leq_* g \Leftrightarrow \forall^\infty n: f(n) \leq g(n)$ nicht-totale Quasiordnung (nicht antisymmetrisch $f\leq_* g \land g \leq_* g \neg\implies f=g$)

\textbf{O-Ordnung}

$f \leq_{\bigOSymbol} g \Leftrightarrow \exists c \in \realnum_+ \forall^\infty n: f(n) \leq c \cdot g(n)$ und $f \lt_{\bigOSymbol} g \Leftrightarrow \forall c \in \realnum_+ \forall^\infty n: f(n) \leq c \cdot g(n)$

O-Notation nicht-totale Quasiordnung

\textbf{Funktionsordnungssätze 14.10}

$f \leq g \implies f \leq_* g$

$f \leq_* g \implies f \leq_{\bigOSymbol} g$

Es gibt $f \leq_* g$ sod. nicht $f \leq g$

Es gibt $f \leq_{\bigOSymbol} g$ sod. nicht $f \leq_* g$

\textbf{Rechenregeln O-Notation}

Falls $f \leq_{\bigOSymbol} h$ und $g \leq_{\bigOSymbol} h$. Dann gilt $f+g \leq_{\bigOSymbol} h$.

Für $c \in \realnum_+$ gilt, falls $f \leq_{\bigOSymbol} g$, so $c \cdot f \leq_{\bigOSymbol} g$.

\textbf{Kardinalvergleiche O-Notation}

Für $c,d \in \realnum_+$ mit $d \gt c$ gilt $\left(n^c\right)_{n \in \natnum} \lt_{\bigOSymbol} \left(n^d\right)_{n \in \natnum}$.

Für $c \in \realnum_+$ und $a \in \realnum_{\gt 1}$ gilt $\left(n^c\right)_{n \in \natnum} \lt_{\bigOSymbol} \left(a^n\right)_{n \in \natnum}$.

Für $a,b \in \realnum_{\gt 1}$ mit $a \lt b$ gilt $\left(a^n\right)_{n \in \natnum} \lt_{\bigOSymbol} \left(b^n\right)_{n \in \natnum}$.

\end{multicols}

\end{document}

